\chapter{Software Listings}

\section{Computation of BBCEAS}\label{sec:comp_bbceas}


With the \ac{BBCEAS} and error equations on hand, the complete computation of
$\alpha$ can now be performed. It is summarised as follows, assuming that 50
acquisitions of $I$ and $I_0$ were taken, along with 50 images of just the
background \ac{CCD} noise.

\begin{enumerate}
  \item Average the \ac{CCD} noise images together to acquire one image $z$
        of the noise on the \ac{CCD}.
  \item For the first image in $I_0$, use a blob detection algorithm to find
        a bounding box $b$ for the signal on the \ac{CCD}.
  \item For each image in $I_0$,
    \begin{enumerate}
      \item Subtract the background noise $z$ from the image.
      \item Average the columns of the pixels in $b$ that correspond to each
        wavelength to create an intensity spectrum $I_{0_{i}}$
    \end{enumerate}
  \item With a set of intensity spectra $I_{0_i}$, calculate the mean value
        $I_0$ and the standard deviation $\sigma_{I_0}$.
  \item Perform steps 3-4 for the set of images corresponding to $I$.
  \item Calculate $\alpha$ using equation \eqref{eq:ceas_geo} and
    $\sigma_\alpha$ using equation \eqref{eq:ceas_err_geo}.
\end{enumerate}

There are two points to note over how most \ac{BBCEAS} calculations are done.
The first is that the acquisition of several intensity spectra is required to
calculate the final error in $\alpha$.

The second interesting addition is the use of the blob detection algorithm.
This modification over a simple cropping of the image allows multiple
detectors (or for a detector to be modified) while still using the same
algorithm to analyse the results. Additionally the blob detection
automatically removes sections of low light intensity from the images, which
are areas that are prone to divide by zero errors and high standard deviations
in $\alpha$.
