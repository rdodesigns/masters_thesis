\chapter{Calibrations}\label{ch:cal}

\acl{BBCEAS} has a simpler setup and lower cost in comparison to \ac{CRDS};
however \ac{BBCEAS} does require a few calibration procedures in order to
acquire accurate absorption spectra. The basic calibrations that should be
done with a \ac{BBCEAS} instrument are: a measurement of the absorption limits
where the signal is linear, an analysis of the error of the detected intensity
signal, and the reflectivity of the mirrors used for the optical cavity.

To find the absorption coefficient range where \ac{BBCEAS} experiment results
follow a linear correlation with absorber concentration, it is common to test
the instrument using a strong absorber. One such absorber is rhodamine 6G,
a dye that can be used to create a laser \cite{Pappalardo:1970hi} or as a
fluorescent marker \cite{Gear:1974tf}.

The error in the detection of the intensity comes down to two important
sources: the error in the \ac{CCD} and the intensity fluctuation over time of
the light source. The intensity fluctuations are especially important, as it
is possible -- and in some cases, extremely likely -- for the intensity of
the light entering the cavity to change between the measurement of the blank
and the measurement of the sample. This fluctuation at best provides a false
understanding of the detection limit of the setup and, in the case of high
variability in the input source, silently modulates the absorption peaks in
acquired spectra.

Finally, the mirror reflectivity is required for any of the equations of the
absorption coefficient or its standard deviation. Using the average mirror
reflectivity across the entire wavelength range of a spectrum can lead to
inaccurate absorption coefficients and alter the shape of the calculated
spectrum \cite{Berden:2009wk}. This case is more common than one may expect,
as the multiple layers of thin films deposited on the mirror's surface
introduces characteristic ripples in the mirror reflectivity as a function of
wavelength due to thin film interference \cite{Islam:2007ea}.

This chapter will discuss these calibrations for the stated problems to
characterise the \ac{BBCEAS} instrument used in this report.



\section{Limits of linearity using Rhodamine 6G}\label{sec:rhodamine}

There are two important considerations for regions where the concentration
of rhodamine correlates linearly with the concentration of an absorber. One
consideration is the calculation of absorption using $A = \alpha l$, where $l$
is the path length when considering multiple passes through the cavity. It is
possible to measure an $\alpha$ value that leads to an absorption of greater
than one, which has no physical meaning.

\begin{figure}
\begin{center}
  \includegraphics[]{figures/plots/rh6g/rh6g.pdf}
\end{center}
\caption[Rhodamine 6G Absorption Spectra as a Calibration]{Absorption spectra of rhodamine 6G at several concentrations. The filled area around each line represents the standard deviation of the signal, calculated using equation \eqref{eq:ceas_err_geo}. The flaring of the error signal near 600\,nm is a result of significantly less light reaching the cavity due to the 585\,nm edgepass filter in the optical setup (Figure~\ref{fig:optical_layout}). The spectral resolution is 4\,nm.}
\label{fig:rh6g}
\end{figure}


While the upper bound to the linearity of the \ac{BBCEAS} equation is useful
for determining when experimental results may be invalid, a tighter bounding
set is useful to determine what sorts of concentrations would be best
measured by the \ac{BBCEAS} technique. Unfortunately, the only way to measure
where absorption coefficients correlate linearly with concentration is to
acquire spectra from many different concentrations and plot the absorption
at the peak wavelength as a function of concentration. This is shown in
Figure~\ref{fig:rh6g_lin}.

\begin{figure}[th]
\begin{center}
  \includegraphics[]{figures/plots/rh6g_linear/rh6g_linear.pdf}
\end{center}
\caption[Dynamic Range of Rhodamine 6G Measurements with \ac{BBCEAS}]{The linear region (left) and nonlinear region (right) of the absorption spectra obtained for different Rh6G concentrations, with red coloured linear and cubic fits, respectively. The error bars represent one standard deviation from the observed signal taken from the maximum peak of the Rh6G spectra. Large fluctuations are due to the laser intensity and potentially dimerisation of the rhodamine 6G molecules.}
\label{fig:rh6g_lin}
\end{figure}

As can be seen in figure~\ref{fig:rh6g_lin}, the lower the concentration,
the better a linear regression fits to the data acquired. As such, a lower
bound is better defined as the limit of detection, instead of attempting to
extrapolate a linear function to below the noise levels of the instrument.

One problem with absorption spectroscopy techniques is that the dynamic range
is mainly dependent on the absorber. Strong absorbers will have a smaller
dynamic range due to the larger change in intensity at each concentration
step. Weak absorbers have the advantage of greater detection ranges under
this regime, but suffer from a lower sensitivity to the concentration as
small changes in intensity correlate to larger changes in concentration, in
comparison to strong absorbers.

It is possible to define the limits where the signal to concentration
are linear by calculating an approximate dynamic range based on a known
absorber. Unfortunately, many substances of interest have unknown or poorly
characterised absorption spectra in terms of molar emissivity in the liquid
phase, and there are no convenient methods of determining valid concentration
ranges.



\section{Intensity Fluctuations in BBCEAS measurements}\label{sec:light_fluc}



\subsection{Intensity fluctuations due to light source stability}\label{subsec:laser_fluc}

\begin{figure}[t]
\begin{center}
  \includegraphics[]{figures/plots/intensity_fluc/intensity_fluc.pdf}
\end{center}
\caption[Intensity fluctuations through blank in \ac{BBCEAS} setup]{Fluctuations in the laser intensity over time at 575\,nm, passed through the \ac{BBCEAS} setup with a water blank in the cuvette. Low and high frequency fluctuations due to laser drift are present, as well as the high frequency fluctuations due to the \ac{CCD}. The \ac{CCD} accounts for an intensity fluctuation of approximately 0.5\%.}
\label{fig:laser_fluc}
\end{figure}


\ac{BBCEAS} measurements, while based on \ac{CRDS}, do not share the benefit
of being immune to intensity fluctuations in the source \cite{Berden:2009wk}.
This is because \ac{CRDS} takes the measurement of the intensity of an
individual pulse of light, whereas \ac{BBCEAS} measures the average steady
state ring down intensity, which represents an average of the light intensity
ring down times across multiple pulses (for pulsed wave sources) or
effectively infinite pulses (for continuous wave sources).

Figure~\ref{fig:laser_fluc} illustrates the problem of light source
fluctuation. In this figure, the fluctuation in each ``bundle'' is
approximately the noise from the \ac{CCD}. By making this assumption, two
different phenomena can be seen in this figure.

The first problem is that the intensity fluctuates in both a high frequency
range (mostly the \ac{CCD} noise) and a lower frequency. This lower frequency
is a result of the drift in the intensity of the output of the supercontinuum
laser. As such, it is simple to see that even a measurement within five
minutes of the blank sample is prone to intensity fluctuation error.

The second problem with the laser fluctuation is that while most of the
high frequency noise is due to the \ac{CCD}, the stretching of the bundles
suggests that there is a high frequency noise component of the supercontinuum
source as well. This means that the fluctuation of the light source skews the
distribution due to the \ac{CCD} noise, sometimes broadening the width (such
as in the second, third and eighth bundles) and sometimes compressing the
noise distribution (as seen in the second to last bundle).

Combined, these two effects of the intensity fluctuation of the laser lead
to errors in calculating $\alpha$ due to drift between measurements and
$\sigma_{\alpha}$ due to fluctuations in $\sigma_{I}$ and $\sigma_{I_0}$. An
additional error arises when considering that these fluctuations in intensity
are often wavelength dependent, so it is not simple to extrapolate the error
calculations in intensity from one wavelength to another.

If the blank and sample spectra are taken within a few minutes of each other,
these error effects are minimised, but still lead to unaccountable error in
the final absorption spectrum. Given that the calculation of the standard
deviations will take part of the high frequency noise into account, and
the fast acquisition will minimise the low frequency source of noise, most
measurements can tolerate these sources of error. For a better sensitivity,
information about the light source fluctuation over time is required to remove
the laser noise. Without this information, it is impossible to give a true
estimate of the sensitivity of a \ac{BBCEAS} instrument.



\subsection{Intensity fluctuations as a result of turbulence}

\begin{figure}
\begin{center}
\includegraphics[]{figures/plots/water_relax/water_relax.pdf}
\end{center}
\caption[Intensity during Turbulence Fluid Injection]{Intensity signal through the \ac{BBCEAS} setup after injecting water into the cuvette that contained water initially. The return to the original signal strength fits an exponential trend well, with a time constant $\tau \approx 13\,\text{s}$.}
\label{fig:relax}
\end{figure}

An additional source of error that causes intensity fluctuations is, perhaps
surprisingly, the turbulence of the solution. This can be clearly seen in
Figure~\ref{fig:relax}, where the intensity detected fluctuates wildly during
the injection of liquid into the cuvette. After the injection phase the
intensity value reaches a maximum, falls and then slowly builds back up to the
beginning level.

The fluctuation and slow build up is due to the alignment of the light within
the cavity. Any optical cavity is very sensitive to deflections of any sort.
The turbulence in the cuvette deflects the path of the light slightly.
This disrupts the measured intensity by changing the total build up intensity
of light within the cavity and by changing the coupling efficiency of the
resulting light with the fibre based grating spectrometer.

Once the injection of liquid has ceased, the resulting intensity build
up occurs as an exponential rise back to the default value. From this
information, the time required to wait after an injection for this particular
cavity size is approximately one minute. Knowing this time constant, and the
slow fluctuation of the light source over time, it is possible to predict a
time window in which the highest quality, lowest noise spectra can be taken.



\section{Mirror Reflectivity}\label{sec:mirror_considerations}

\begin{figure}
\begin{center}
  \includegraphics[]{figures/plots/mirror_reflectivity/mirror_reflectivity.pdf}
\end{center}
\caption[Cavity Mirror Reflectivity]{The mirror reflectivity provided by the manufacturer. The cavity alignment has a direct effect on the mirror reflectivity seen in an experiment, and hence these curves cannot be used directly.}
\label{fig:mirror}
\end{figure}

No matter what \ac{BBCEAS} equation an experimenter decides to use to calculate
the absorption spectrum of a sample, the mirror reflectivity must be explicitly
known, as it appears in the form of $\tfrac{1-R}{d}$ \eqref{eq:ceas_std} or in
multiple places for other equations \eqref{eq:ceas_geo_mod}. The inclusion of
$R$ in these equations represents a correction to the path length to account
for the number of passes that light undergoes once it is inside the cavity.

Neglecting to determine the mirror reflectivity of the cavity, as a function
of wavelength, leads to two consequences. The simplest is that the calculated
values of absorption can have an extra error that can easily exceed error
due to other sources. While a calibration against a known concentration it
is possible to neglect this source of error when attempting to determine
the unknown concentration of an absorber, the values calculated provide an
inaccurate understanding of the actual absorption of a particular electronic
transition.

The second, more peculiar result of neglecting the mirror reflectivity curve is
that the calculated absorption spectrum changes in \emph{shape}, which  can
lead to both artifacts in the spectrum as well as an alteration of the shape of
a transition. Attempting to fit an accurate profile to a transition in a
spectrum would be an difficult at best.

There are two ways to determine the mirror reflectivity curves. The first is
to simply have the manufacturer provide the information, or to digitalise a
graph provided through an \ac{FTS} type measurement, although this is known
to not work very well in practice as the mirror reflectivity is a function of
the alignment of the cavity \cite{Berden:2009wk}. The second common method is
to use the \ac{BBCEAS} setup in a \ac{CRDS} experiment, which, due to the self
calibrating nature of \ac{CRDS}, allows the experimenter to extract the mirror
reflectivity curves.



\section*{Chapter Review}

While \ac{BBCEAS} does require a few calibration steps to acquire a full
understanding of the spectra it produces, these procedures are not unlike
what an experimenter would go through for a normal absorption, single pass
measurement. In addition, these considerations increase the information of
absorption spectra by characterising the noise inherent in the instrument at
each wavelength measured, correcting for the effects that can occur during a
measurement.
