\chapter{Broadband Cavity Enhanced Absorption Spectroscopy}


\ac{BBCEAS} is a recently devised experimental technique that uses the
cavity enhancement factors of \ac{CRDS} without requiring fast and sensitive
detectors to determine the ringdown time. In this chapter we will discuss
how \ac{BBCEAS} works, its advantages and limitations, and the design of the
algorithms and optical system used in this report.


\section{How Broadband Cavity Enhanced Absorption Works}

\acl{BBCEAS} setups very similar to their \ac{CRDS} brethren. In both
cases, light that enters the cavity passes through the sample a certain number
of times (defined according to the reflectivity of the mirrors and the
absorption of the sample) before exiting, to be detected. One
paper\marginpar{Engeln et al. Review Scientific Instruments 1998} proved that,
for a continuous wave input, the intensity of the light transmitted through
the cavity is directly proportional to the ring down time inside the cavity.
As such, one can derive the following function for the absorption coefficient
from the ring down equation.

  \begin{align}
    \alpha(\lambda) =
    \left(\frac{I_0(\lambda)}{I(\lambda)}-1\right)\left(\frac{1-R(\lambda)}{d}\right)
  \end{align}

To take full advantage of this system, \ac{BBCEAS} uses a broadband light
source to acquire information about the absorption of a solution at many
wavelengths simultaneously. This can be done through an incoherent source
-- called \ac{IBBCEAS} -- or with a coherent supercontinuum laser
source -- called \ac{SC-CEAS}.

\subsection{Advantages}

\acl{BBCEAS} has several advantages over its sibling techniques. Regular
\ac{CRDS} or \ac{BBCRDS} require expensive light and detector sources to
accurately acquire ring down signals. This cost and complexity does not
exist in \ac{BBCEAS}, as modest spectrometers and light sources can be used.
Additionally, this technique can be used to acquire high bandwidth spectrum,
limited only by the mirror dielectric coatings.

\ac{BBCEAS} shares some of these advantages against other experimental
techniques as well. Multipass absorption cells are difficult to set up due to
alignment errors, and \ac{PAS} is difficult to use due to the use of a lock-in
amplifier.

Other broadband techniques exist that provide a higher spectral
resolution, but require a laser source such as a \acl{TDL}. These require
special driver circuitry and cannot capture the entire bandwidth in one shot.

\subsection{Limitations}

While there are many benefits to \ac{BBCEAS} for acquiring absorption spectrum
of a solution, the technique trades simplicity and broadband acquisitions for
higher error, lower limit of detection and potentially low spectral
resolution.

The absorptivity resolution of the technique (described in section
\ref{sec:ceas_error}) is limited by the noise of the detector. Most
\ac{BBCEAS} setups use a \ac{CCD} camera to detect the intensity of light that
has passed through a diffraction grating. \acp{CCD} do not have the ability to
detect some of the small differences in light intensity due to the inherent
noise of the detector, unless a cooled \ac{CCD} is used.

Additionally \acp{CCD} are not as sensitive as the photodiodes or \acp{PMT}
used in \ac{CRDS}, which limits the ability to detect minute changes in
concentration. This detection sensitivity can be offset by the use of an
\ac{EMCCD}, but these detectors are expensive at the present time, which
defeats part of the original advantages of \ac{BBEAS}. Even if one was to use
an \ac{EMCCD}, the sensitivity of the setup is unlikely to be that accurate
due to fluctuating light sources and, in some published papers, unknown
reflectivity curves.

While \ac{TDL} based broadband techniques have longer acquisition
times, they easily outperform most \ac{BBCEAS} techniques in terms of spectral
resolution. This is because the narrow linewidths of a \ac{TDL} can be
accurately known and imaged onto a photodiode, whereas the spectrometer in a
\ac{BBCEAS} is limited in resolution by the grating and, more importantly, the
focusing optics and irises used in the spectrometer. It is not uncommon to see
a high resolution spectrum taken with a \ac{TDL} having a spectral resolution
lower than 0.1nm, whereas a grating spectrometer has a spectral resolution on
the order of nanometers.

Another point of contention is the fact that this requires a calibration,
leading to higher error rates.\marginpar{Flesh out calibration idea.}

Finally, without the use of a device such as a \ac{CCD} streak camera, the
temporal resolution is limited by the exposure time of the \ac{CCD}. It would
be terribly difficult to measure the change in a chemical reaction that
occurred on the order of milliseconds using a \ac{CCD}, whereas this type of
measurement would be trivial to watch with the nanosecond temporal resolution
of the \acp{PMT} used in \ac{CRDS}.

\subsection{Considerations for BBCEAS in Liquid Medium}

The advantages and limitations of a technique must be put into context
to determine its value for a particular experimental setup. In
chapter~\ref{chap:calibration} and \ref{chap:europium} the \ac{BBCEAS}
setup laid out in this chapter is used to detect the absorbers in liquid
solutions. As such, the low spectral resolution obtained with a modest grating
spectrometer is good enough for the wide features present in liquid solutions
(with rare exceptions, noted in section~\ref{sec:expect_theory_results}).
Additionally, the samples measured in this report do not vary over time, and
hence the low temporal resolution is unimportant.

\section{Algorithms for BBCEAS}

The basic procedure for a CEAS experiment is to do the following.

\begin{enumerate}
  \item Setup the instrument and perform all of the alignment with a blank
        inside the sample cuvette.
  \item Perform a measurement of the intensity per wavelength leaving the
        cavity with the blank inside the cuvette. This can be done a number of
        times $n$ to acquire a mean intensity $I_0$ and its standard deviation
        $\sigma_{I_0}$
  \item Perform the same measurement with the absorber inside the cuvette,
        acquiring the mean intensity $I$ and its standard deviation
        $\sigma_{I}$.
\end{enumerate}

Once this information has been gathered, one uses the \ac{BBCEAS} equation to
calculate the absorption coefficient $\alpha(\lambda)$, usually in cm$^{-1}$.
However, there are several ways of performing this calculation, as described
below. \marginpar{Give a more detailed output in the appendix, including a
source code listing}

%% Include in appendix
%\subsection{Calculation of Intensity}

%In a \ac{CCD} based grating spectrometer, the light falling on the \ac{CCD}
%will not be on a single row of pixels, but rather create a rectangle where the
%intensity signal becomes fainter the farther one measures from the rectangle's
%long center line.

\subsection{BBCEAS Equation}

While all \ac{BBCEAS} setups perform experiments in the same manner, where an
intensity spectrum of a blank and with the solute are taken, there are four
different equations in use in the literature to calculate absorption
coefficient values. The most common one is below, derived from
O'Keefe.\marginpar{For all equations in this section, be careful of when $I =
0$.}

  \begin{align}
    \alpha(\lambda) = \left(\frac{I_0(\lambda)}{I(\lambda)}-1\right)\left(\frac{1-R(\lambda)}{d}\right)\label{eq:ceas_std}
  \end{align}

\marginpar{$\lambda$ dependence is implied for $\alpha$, $R$, $I_0$, and $I$ in the following derivations}
This equation is a result of the cavity ring down equation at steady state,
under the assumptions that the mirror reflectivity $R$ tends to 1 and that the
absorption coefficient tends to nearly zero. Due to these assumptions, this
technique is best served for the analysis of gas phase solutions, although it
has been shown to be applicable over almost all experimental conditions that
one would want to run.\marginpar{At least all experimental conditions in the
literature}

A second equation in use in the literature does not make the assumptions
that $R \to 1$ and $\alpha \to 0$ due to the fact that it is derived from a
geometric sum instead of an indefinite integral. The equation is slightly more
complicated.

  \begin{align}
    \alpha = \frac{1}{d}\left|\ln\left(\frac{1}{2R^2}\left(\sqrt{4R^2+\left(\frac{I_0}{I}(R^2-1)\right)^2} + \frac{I_0}{I}(R^2-1)\right)\right)\right| \label{eq:ceas_geo}
  \end{align}

While this equation appears to be unwieldy, it is not more computationally
complex than the standard equation. Because the difference in using the
equations is only the time it takes to write into a programming language,
one can choose to use either equation. However, the lack of assumptions for
the geometrically derived \ac{BBCEAS} equation makes it a better candidate
for analysis since one does not have to worry about the cases where the
mirror reflectivity is low (such as those that would be used in a low cost
instrument) or for high absorption coefficients.

\subsection{Error calculation}\label{sec:ceas_error}

\marginpar{See appendix for full error calculation derivation.}

The standard way to calculate the minimum level of detection (as well as the
error) in a \ac{BBCEAS} is to use the following equation, which applies for
calculating the error from equation \eqref{eq:ceas_std}.

\begin{align}
  \alpha_{\text{min}} = \left(\frac{\Delta
  I}{I_0}\right)\left(\frac{1-R}{d}\right)\label{eq:ceas_min}
\end{align}

In this equation, $\Delta I$ is the minimum detectable change in $I$.
Similarly, the limit of detection (LOD) is normally defined as three times the
standard deviation in $\alpha$.
\begin{align}
  \text{LOD}_{\alpha} = 3\sigma_{\alpha}\label{eq:lod}
\end{align}

These equations present two different problems.

One is the way in which the minimum detectable change is defined. If the
minimum detectable change is defined as how confidently one can assume a new
intensity value is unique to an absorber, one would likely use something
similar to the limit of detection. Alternatively, one could define the minimum
detectable change in intensity as a function of the noise in the detector
only, instead of through the entire experiment (as one assumes all other
sources to be constant).

While the photodetector is often the main source of noise, this kind of
analysis excludes the intensity fluctuations in the light source in
calculating $\alpha$. As such, this error is too generous to the actual
achievable sources of error in a \ac{BBCEAS} experiment.

The second problem with the error calculations is that the error in the blank
measurement is not taken into account. Equation \eqref{eq:ceas_min} assumes
that one has a constant for the intensity of the blank $I_0$. However, we know
that this blank has its own distribution of points. As such, the error
equation in the literature is generous in the minimum detectable absorption
changes.

In an actual experiment, it may be hard to control the intensity distributions
to be stable over time. As such, one cannot make assumptions about the width
of the intensity distribution measurements accurately.

To alleviate the problems of the error associated with the intensity of the
source and that of the initial blank intensity measurement, one may consider a
more accurate method of attempting to calculate the absorption coefficient
many times. However, this can be immensely time consuming to acquire a
statistically accurate dataset for many concentrations, and requires many
flushes of the liquid inside the cuvette (increasing the sample volume
costs).

\ac{BBCEAS} experiments can calculate a more accurate error by altering the
method of spectrum collection. When an experimenter collects data on the
intensity distribution of the blank and solution samples, they often take many
snapshots of this information and then average that data. However, if one
calculates the standard deviation of the intensity signals, then it is
possible to \emph{approximate} the standard deviation in $\alpha$ using the
delta method. This leads to the following equations of error for equation
\eqref{eq:ceas_std} and \eqref{eq:ceas_geo}, respectively.

    \begin{align}
      \sigma_\alpha = \left(\frac{\sigma_{I_0}}{I} +
      \frac{I_0\sigma_I}{I^2}\right)
      \left(\frac{1-R}{d}\right)\label{eq:ceas_err_std}
    \end{align}


    \begin{align}
      \sigma_\alpha = \left(\frac{\sigma_{I_0}}{I} +
      \frac{I_0\sigma_I}{I^2}\right)
      \left(\dfrac{R^2-1}{d}\right)\left(4R^2+\left(\frac{I_0}{I}(R^2-1)\right)^2\right)^{-1/2}\label{eq:ceas_err_geo}\\\text{
      given $\alpha \neq 0$}\notag
    \end{align}

It is possible to calculate the standard deviation in $\alpha$ exactly, as a
measure of sanity, by comparing each of the intensity spectrum of the
blank against each of the intensity spectrum of the sample. However, this
quickly becomes computationally demanding. If one was to take 50
intensity spectrum of the blank and the sample, over 1024 pixel bins inside a
CCD camera, then one would calculate $\alpha$ more than 2.5 million times,
which would take a while even on computers from 2012.

\subsection{Notes on calculations}

While the equations for \ac{BBCEAS} and the standard deviation of the signals
is relatively straight forward, there are bounds on the calculation of
$\alpha$ that are implied.

The reader will probably notice that equation \eqref{eq:ceas_err_geo} comes
with the tag that $\alpha \neq 0$. In the derivation, this is because when
$\alpha = 0$, the error $\sigma_{\alpha}$ is also zero due to attempting to
perform the derivative of an absolute value. However, even without this
limitation, an experimenter should ignore $\alpha = 0$, as this indicates
that the intensity measurements were indiinguishable. At such a value, a more
meaningful result would be to state that the absorption is insignificant with
reference to the LOD$_{\alpha}$.

Another problem is when the measured intensity $I$ is zero. In all of the
equations stated thus far result in a division by zero when the intensity
value is zero. The optimistic reader may state that this would never be
the case, as there will always be background counts on the detector due to
\ac{CCD} noise. However $I$ can be zero if one subtracts the background
\ac{CCD} noise from the acquired images. This subtraction is required for an
accurate measurement of $\alpha$.

This phenomenon will occur at high absorptivity values, and due to the nature
of pressure and collision broadening means that there is a high probability
that several points in a row in the intensity spectrum will be zero or nearly
zero. The resulting $\alpha$ spectrum would have a peak where the top was cut
off. While this scenario is not ideal, there is a way acquire an approximate
value for the $\alpha$ values that have been cut.

The line shapes of an electronic transition (assuming that there are no other
significant absorptions near a certain transition), after the convolution of
collision broadening, pressure broadening, and the broadening associated with
the spectrometer, take on a voigt profile. Since these broadening effects
impact all transitions equally, we can use another peak within the absorption
spectrum to calculate the parameters for this profile. It is then trivial to
scale the voigt profile to the peak that has been cut off at the top and
arrive at an approximate value of $\alpha$. The only potential problem with
this method is that all of the existing methods shown for calculating the
error in the absorption coefficient are inapplicable.

\subsection{Computation}

\marginpar{The complete algorithm can be found in the appendix. However, a
brief version can be found below.}

With the \ac{BBCEAS} and error equations on hand, the complete computation of
$\alpha$ can now be performed. It is summarised as follows, assuming that 50
acquisitions of $I$ and $I_0$ were taken, along with 50 images of just the
background \ac{CCD} noise.

\begin{enumerate}
  \item Average the \ac{CCD} noise images together to acquire one image $z$
        of the noise on the \ac{CCD}.
  \item For the first image in $I_0$, use a blob detection algorithm to find
        a bounding box $b$ for the signal on the \ac{CCD}.
  \item For each image in $I_0$,
    \begin{enumerate}
      \item Subtract the background noise $z$ from the image.
      \item Average the columns of the pixels in $b$ that correspond to each
        wavelength to create an intensity spectrum $I_{0_{i}}$
    \end{enumerate}
  \item With a set of intensity spectra $I_{0_i}$, calculate the mean value
        $I_0$ and the standard deviation $\sigma_{I_0}$.
  \item Perform steps 3-4 for the set of images corresponding to $I$.
  \item Calculate $\alpha$ using equation \eqref{eq:ceas_geo} and
    $\sigma_\alpha$ using equation \eqref{eq:ceas_err_geo}.
\end{enumerate}

There are two points to note over how most \ac{BBCEAS} calculations are done.
The first is that the acquisition of several intensity spectra is required to
calculate the final error in $\alpha$.

The second interesting addition is the use of the blob detection algorithm.
This modification over a simple cropping of the image allows multiple
detectors (or for a detector to be modified) while still using the same
algorithm to analyse the results. Additionally the blob detection
automatically removes sections of low light intensity from the images, which
are areas that are prone to divide by zero errors and high standard deviations
in $\alpha$.


\section{Optical Layout and Design}

The optical layout for the system used in this report is nearly identical to
all \ac{BBCEAS} layouts. The light source is a Fianium \textsc{sc-450} supercontinuum
pulsed laser source (with a high enough repetition rate that it is treated
like a continuous wave laser source in our equations). After the source, the
cavity uses cavity mirrors of reflectivity $R=0.99$. Finally, light exiting
the cavity is coupled into a multimode fibre and imaged using a custom
built grating spectrometer (discussed below).\marginpar{Include image of
optical setup}
\marginpar{Should I include the mirror reflectivity curve as an image?}

\subsection{Spectrometer}
\marginpar{Should I put this in an appendix?}

The spectrometer used is a simple transmission grating spectrometer
with a fibre input and a standard \ac{CCD} from a digital
consumer camera.\marginpar{include spectrometer diagram}


\section{Potential Enhancements}

While this setup works well, there are certain aspects of the instrument that
can be made to make either a gain in sensitivity and/or in cost over the
current implementation.

\subsection{LED source}

The supercontinuum light source is excellent for its low divergence and high
power properties over the visible range. However, such light sources are far
from cheap, and supercontinuum sources often have high fluctuations in
intensity over time that can be difficult to correct for without a reference
signal.

\acp{LED} offer an enticing visible light source alternative to supercontinuum
sources on the basis of their exceptionally low cost and stable output
intensity stability.

The current costs of high power \ac{LED} devices are on the order of only
a few pounds, and can be easily wired in series with a resistor in order
to provide stable light output. There is a potential complication in the
power supplies for high powered \acp{LED} because of power draws of over 10
watts (usually requiring over 10 volts and 1 ampere to create a stable power
source). However, one can use a boost-buck style DC to DC power converter to
create highly stable current inputs into the \ac{LED} to aid in stabilising
the intensity output. Additionally a thermistor can be added to the \ac{LED}
to create a closed loop system, where the feedback from the thermistor assists
in stabilising the light source over time.

Once a stable output intensity is reached, the light from an \ac{LED} can be
fibre coupled into a multimode fibre. The light coming out of the fibre can
then be collimated to create a low divergence beam to be put into the cavity.
Other groups have shown that fibre-coupled \ac{LED} light sources can be used
for \ac{BBCEAS} experiments, and that the setup for an \ac{LED} source is
relatively trivial. It would seem the stability can be improved in comparison
to published research by using a closed feedback loop for the \ac{LED}.

An additional benefit of an \ac{LED} based light source is that the intensity
could be easily modulated by an external signal quickly. This could be used to
create lock-in type detection schemes, or to initiate a photochemical reaction
and watch the result through absorption spectroscopy.\marginpar{Clearly other
possibilities exist to use a modulated input signal as well}

\subsection{Non-fibre Spectrometer}

Coupling light into a multimode fibre can provide a convenient way to reroute
light to certain optical setups. The problem with coupling light into a fibre,
even a multimode fibre with a large diameter core, is that the loss of light
at the interface of the fibre is relatively high. Practically, lasers are
unaffected by this procedure in \ac{BBCEAS} measurements as it is often
possible to reach higher output power levels to offset the coupling losses.
There are even cases where one can saturate the absorber in a \ac{BBCEAS}
measurement if the light intensity is too high.

Incoherent light sources such as \acp{LED}, on the other hand, suffer from
problems of having wide, slightly divergent beam profiles, which practically
means that these sources have limits on their intensity that cannot offset the
fibre losses. On top of this limit, losses for the incoherent source are
higher than the laser due to the larger spot sizes and diffuse nature of the
light when coupling into a fibre.

The current optical setup is convenient because different detectors can be
easily swapped to provide more accurate or faster acquisitions. But tests with
collimated \ac{LED} light sources has required integration
times on the order of ten seconds, hampering the effective use of \ac{LED}
sources.

If the light coming out of the cavity is instead directed onto a transmission
grating, the losses due to the coupling would be removed, and much more light
would hit the detector. The loss of convenience would be greatly offset by the
ability acquire data at a faster rate and lower noise than in the case of the
fibre coupled spectrometer.

\subsection{Webcam}

While the current system uses a \ac{CCD} that costs much less than the
detectors in other \ac{CEAS} and \ac{CRDS} systems, the Starlight is still on
the order of a thousand pounds to buy. To further reduce the costs of the
system, the detector can be replaced with a \ac{CCD} from a web camera, which
costs tens of pounds.

Replacing the \ac{CCD} with a web camera would bring the total costs, if one
uses low reflectivity dielectric mirrors, to potentially less than a hundred
points. The \ac{CCD} in a web camera is also incredibly tiny, and would allow
the entire \ac{BBCEAS} setup to be easily miniaturised into a handheld device
using an \ac{LED} light source.\marginpar{Why would we use a webcam instead of
a line detector for a handheld device? We are not using the 2D aspect of the
pixels....}

\section{Summary}
Review Chapter Here.
